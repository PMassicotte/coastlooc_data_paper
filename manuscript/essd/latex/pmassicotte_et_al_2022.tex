%% Copernicus Publications Manuscript Preparation Template for LaTeX Submissions
%% ---------------------------------
%% This template should be used for copernicus.cls
%% The class file and some style files are bundled in the Copernicus Latex Package, which can be downloaded from the different journal webpages.
%% For further assistance please contact Copernicus Publications at: production@copernicus.org
%% https://publications.copernicus.org/for_authors/manuscript_preparation.html


%% Please use the following documentclass and journal abbreviations for preprints and final revised papers.

%% 2-column papers and preprints
\documentclass[essd, manuscript]{copernicus}

%% Journal abbreviations (please use the same for preprints and final revised papers)


% Advances in Geosciences (adgeo)
% Advances in Radio Science (ars)
% Advances in Science and Research (asr)
% Advances in Statistical Climatology, Meteorology and Oceanography (ascmo)
% Annales Geophysicae (angeo)
% Archives Animal Breeding (aab)
% ASTRA Proceedings (ap)
% Atmospheric Chemistry and Physics (acp)
% Atmospheric Measurement Techniques (amt)
% Biogeosciences (bg)
% Climate of the Past (cp)
% DEUQUA Special Publications (deuquasp)
% Drinking Water Engineering and Science (dwes)
% Earth Surface Dynamics (esurf)
% Earth System Dynamics (esd)
% Earth System Science Data (essd)
% E&G Quaternary Science Journal (egqsj)
% European Journal of Mineralogy (ejm)
% Fossil Record (fr)
% Geochronology (gchron)
% Geographica Helvetica (gh)
% Geoscience Communication (gc)
% Geoscientific Instrumentation, Methods and Data Systems (gi)
% Geoscientific Model Development (gmd)
% History of Geo- and Space Sciences (hgss)
% Hydrology and Earth System Sciences (hess)
% Journal of Bone and Joint Infection (jbji)
% Journal of Micropalaeontology (jm)
% Journal of Sensors and Sensor Systems (jsss)
% Magnetic Resonance (mr)
% Mechanical Sciences (ms)
% Natural Hazards and Earth System Sciences (nhess)
% Nonlinear Processes in Geophysics (npg)
% Ocean Science (os)
% Polarforschung - Journal of the German Society for Polar Research (polf)
% Primate Biology (pb)
% Proceedings of the International Association of Hydrological Sciences (piahs)
% Scientific Drilling (sd)
% SOIL (soil)
% Solid Earth (se)
% The Cryosphere (tc)
% Weather and Climate Dynamics (wcd)
% Web Ecology (we)
% Wind Energy Science (wes)


%% \usepackage commands included in the copernicus.cls:
%\usepackage[german, english]{babel}
%\usepackage{tabularx}
%\usepackage{cancel}
%\usepackage{multirow}
%\usepackage{supertabular}
%\usepackage{algorithmic}
%\usepackage{algorithm}
%\usepackage{amsthm}
%\usepackage{float}
%\usepackage{subfig}
%\usepackage{rotating}

% Packages needed for table 1
\usepackage{longtable}
\usepackage{booktabs}
\usepackage[table]{xcolor}
\usepackage{array}
\usepackage[]{pdflscape}

% For the curvy "l" in the COASTLOOC word
% https://tex.stackexchange.com/questions/573915/how-to-add-a-font-style-for-a-specific-letter-only
\usepackage[scr=boondoxo]{mathalpha}

\begin{document}

\title{TEXT}


% \Author[affil]{given_name}{surname}

\Author[1]{Philippe}{Massicotte}
\Author[]{}{}
\Author[]{}{}

\affil[]{ADDRESS}
\affil[]{ADDRESS}

%% The [] brackets identify the author with the corresponding affiliation. 1, 2, 3, etc. should be inserted.

%% If an author is deceased, please mark the respective author name(s) with a dagger, e.g. "\Author[2,$\dag$]{Anton}{Smith}", and add a further "\affil[$\dag$]{deceased, 1 July 2019}".

%% If authors contributed equally, please mark the respective author names with an asterisk, e.g. "\Author[2,*]{Anton}{Smith}" and "\Author[3,*]{Bradley}{Miller}" and add a further affiliation: "\affil[*]{These authors contributed equally to this work.}".


\correspondence{NAME (EMAIL)}

\runningtitle{TEXT}

\runningauthor{TEXT}

\received{}
\pubdiscuss{} %% only important for two-stage journals
\revised{}
\accepted{}
\published{}

%% These dates will be inserted by Copernicus Publications during the typesetting process.


\firstpage{1}

\maketitle

\begin{abstract}
    Coastal Surveillance Through Observation of Ocean Color (COAST$\mathscr{l}$OOC) oceanographic expeditions were conducted in 1997 and 1998 to obtain a synoptic view of the spatial distribution of different biological, chemical and physical variables across the land-to-sea gradient along the European coasts. A total of 379 stations distributed in six areas were visited: (1) 39 in the Adriatic Sea, (2) 38 in the Atlantic Ocean, (3) 57 in the Baltic Sea, (4) 85 in the English Channel, (5) 61 in the Mediterranean Sea and (6) 99 in the North Sea. A particular emphasis has been dedicated to the collection of a comprehensive set of apparent (AOPs) and inherent (IOPs) optical properties to document carbon fluxes at both the local and global scales. These radiometric quantities have been measured using traditional ship-based sampling, but also from a helicopter in shallow estuaries, which are more difficult to access from boats. Although that the COAST$\mathscr{l}$OOC campaigns were carried out more than 20 years ago, the rich and historical dataset that has been collected has great potential to contribute to the development and evaluation of new bio-optical models adapted for optically-complex waters. Given that this unique dataset is still today frequently requested by other researchers, we present the result of an effort to compile and standardize data that will facilitate their reuse in other oceanographic studies. The dataset is available at https://doi.org/10.17882/75345 (Massicotte2022).
\end{abstract}

\copyrightstatement{TEXT} %% This section is optional and can be used for copyright transfers.


\introduction  %% \introduction[modified heading if necessary]

Since the launch of the Coastal Zone Color Scanner (CZCS) by NASA in 1978, ocean color remote sensing has been used to monitor the state and the evolution of global marine ecosystems both in time and space. In open oceans, the main component that affects the variations in the inherent (IOPs) and apparent (AOPs) optical properties of seawater is phytoplankton, which is usually represented by the concentration of chlorophyll-a (\citep{Morel1977}). Many simple empirical spectral band ratio algorithms have been developed to link changes in remotely-sensed ocean color (OC), measured as reflectance, to the variations in chlorophyll-a concentration (see OReilly2019 for an extensive evaluation of OC band ratio algorithms). Because these algorithms perform surprisingly well, a plethora of studies have been conducted, notably about phytoplankton phenology (e.g., \citealt{Vargas2009}) and phytoplankton primary production (see \citealt{Carr2006} and references therein).

\section{Study area and sampling overview}
TEXT


\subsection{Study area and general sampling strategys}

During the COAST$\mathscr{l}$OOC campaigns, a total of 420 locations were visited. These locations were spread out along the coasts of the Mediterranean Sea (\textit{n} = 41 in case 1 water, \textit{n} = 61 in case 2 water), Adriatic Sea (n = 39), Baltic Sea (n = 57), North Sea (n = 99), English Channel (n = 85) and Atlantic Ocean (n = 38) Within each area, the stations were generally distributed along across-shore or along-shore transects to capture the land-to-sea gradients and document river plumes (Fig. 1B). Stations were sampled either with a helicopter or a ship between 1997-04-02 and 1998-09-25 (Fig. 2A). Compared to traditional ship-based sampling, the helicopter platform allowed to efficiently sample shallow estuaries, which can be difficult to access by boat (some samples were collected in waters as shallow as 1 m, \citealt{Babin2003}). Combining both ship and airborne sampling approaches allowed covering the whole inshore to open-ocean aquatic continuum. The bathymetry \citep{GEBCO2020} varied greatly across the stations, where it averaged 10 meters in the Adriatic Sea and 2 600 meters in the Case 1 Mediterranean Sea (Fig. 2B).


\subsubsection{HEADING}
TEXT


\conclusions  %% \conclusions[modified heading if necessary]
TEXT

%% FIGURES

%% When figures and tables are placed at the end of the MS (article in one-column style), please add \clearpage
%% between bibliography and first table and/or figure as well as between each table and/or figure.

% The figure files should be labelled correctly with Arabic numerals (e.g. fig01.jpg, fig02.png).


%% ONE-COLUMN FIGURES

\begin{figure}[t]
    \includegraphics[width=12cm]{../../../graphs/fig01.pdf}
    \caption{My caption}
\end{figure}

\clearpage

\begin{figure}[t]
    \includegraphics[width=12cm]{../../../graphs/fig02.pdf}
    \caption{My caption}
\end{figure}

\clearpage

\begin{figure}[t]
    \includegraphics[width=12cm]{../../../graphs/fig03.pdf}
    \caption{My caption}
\end{figure}

\clearpage

\begin{figure}[t]
    \includegraphics[width=12cm]{../../../graphs/fig04.pdf}
    \caption{My caption}
\end{figure}

\clearpage

\begin{figure}[t]
    \includegraphics[width=12cm]{../../../graphs/fig05.pdf}
    \caption{My caption}
\end{figure}

\clearpage

\begin{figure}[t]
    \includegraphics[width=12cm]{../../../graphs/fig06.pdf}
    \caption{My caption}
\end{figure}

\clearpage

\begin{figure}[t]
    \includegraphics[width=12cm]{../../../graphs/fig07.pdf}
    \caption{My caption}
\end{figure}

\clearpage
\begingroup\fontsize{8}{10}\selectfont

\begin{longtable}[t]{>{\raggedright\arraybackslash}p{18em}>{\raggedright\arraybackslash}p{8em}>{\raggedright\arraybackslash}p{10em}>{\raggedright\arraybackslash}p{25em}}
\caption{List of measured parameters}\\
\toprule
Variable & Units & Source file & Description\\
\midrule
\endfirsthead
\caption[]{List of measured parameters \textit{(continued)}}\\
\toprule
Variable & Units & Source file & Description\\
\midrule
\endhead

\endfoot
\bottomrule
\endlastfoot
\cellcolor{gray!6}{wavelength} & \cellcolor{gray!6}{nm} & \cellcolor{gray!6}{100303.csv} & \cellcolor{gray!6}{}\\
\addlinespace
a\_p\_m1 & m\textsuperscript{-1} & 100303.csv & Total particulate absorption\\
\addlinespace
\cellcolor{gray!6}{a\_nap\_m1} & \cellcolor{gray!6}{m\textsuperscript{-1}} & \cellcolor{gray!6}{100303.csv} & \cellcolor{gray!6}{Non-algal absorption}\\
\addlinespace
a\_nap\_adjusted\_m1 & m\textsuperscript{-1} & 100303.csv & Non-algal absorption adjusted so that baseline background is equal to that of ap\\
\addlinespace
\cellcolor{gray!6}{a\_cdom\_m1} & \cellcolor{gray!6}{m\textsuperscript{-1}} & \cellcolor{gray!6}{100303.csv} & \cellcolor{gray!6}{Chromophoric dissolved organic matter absorption}\\
\addlinespace
a\_cdom\_adjusted\_m1 & m\textsuperscript{-1} & 100303.csv & Chromophoric dissolved organic matter absorption with background baseline removed\\
\addlinespace
\cellcolor{gray!6}{a\_phy\_m1} & \cellcolor{gray!6}{m\textsuperscript{-1}} & \cellcolor{gray!6}{100303.csv} & \cellcolor{gray!6}{Phytoplankton absorption}\\
\addlinespace
background\_a\_p\_m1 & m\textsuperscript{-1} & 100303.csv & Baseline background of total particulate absorption\\
\addlinespace
\cellcolor{gray!6}{background\_a\_cdom\_m1} & \cellcolor{gray!6}{m\textsuperscript{-1}} & \cellcolor{gray!6}{100303.csv} & \cellcolor{gray!6}{Baseline background of chromophoric dissolved organig matter absorption}\\
\addlinespace
background\_a\_nap\_m1 & m\textsuperscript{-1} & 100303.csv & Baseline background of non-algal absorption\\
\addlinespace
\cellcolor{gray!6}{a\_m1} & \cellcolor{gray!6}{m\textsuperscript{-1}} & \cellcolor{gray!6}{100304.csv} & \cellcolor{gray!6}{Total non-water absorption coefficient}\\
\addlinespace
c\_m1 & m\textsuperscript{-1} & 100304.csv & Total non-water attenuation coefficient\\
\addlinespace
\cellcolor{gray!6}{bp\_m1} & \cellcolor{gray!6}{m\textsuperscript{-1}} & \cellcolor{gray!6}{100304.csv} & \cellcolor{gray!6}{Particle scattering coefficient}\\
\addlinespace
longitude & Degree decimal & 100305.csv & Longitude of the pixel used to extract the bathymetry\\
\addlinespace
\cellcolor{gray!6}{latitude} & \cellcolor{gray!6}{Degree decimal} & \cellcolor{gray!6}{100305.csv} & \cellcolor{gray!6}{Latitude of the pixel used to extract the bathymetry}\\
\addlinespace
bathymetry\_m & m & 100305.csv & Bathymetry depth at the sampled stations\\
\addlinespace
\cellcolor{gray!6}{eu\_w\_m2\_um} & \cellcolor{gray!6}{W m-2 µm-1} & \cellcolor{gray!6}{100307.csv} & \cellcolor{gray!6}{Upward irradiance just beneath the water surface (Eu0-)}\\
\addlinespace
ed\_w\_m2\_um & W m-2 µm-1 & 100307.csv & Downward irradiance just beneath the water surface (Ed0-)\\
\addlinespace
\cellcolor{gray!6}{k\_eu\_m1} & \cellcolor{gray!6}{m\textsuperscript{-1}} & \cellcolor{gray!6}{100307.csv} & \cellcolor{gray!6}{Attenuation coefficient for upward irradiance (just beneath the water surface)}\\
\addlinespace
k\_ed\_m1 & m\textsuperscript{-1} & 100307.csv & Attenuation coefficient for downward irradiance (just beneath the water surface)\\
\addlinespace
\cellcolor{gray!6}{measured\_reflectance\_percent} & \cellcolor{gray!6}{Percent} & \cellcolor{gray!6}{100309.csv} & \cellcolor{gray!6}{Surface water reflectance}\\
\addlinespace
s\_cdom\_nm1 & nm\textsuperscript{-1} & 100310.csv & Spectral slope that describes the approximate exponential decrease in aCDOM\\
\addlinespace
\cellcolor{gray!6}{s\_nap\_nm1} & \cellcolor{gray!6}{nm\textsuperscript{-1}} & \cellcolor{gray!6}{100310.csv} & \cellcolor{gray!6}{Spectral slope that describes the approximate exponential decrease in aNAP}\\
\addlinespace
a\_cdom443\_m1 & m\textsuperscript{-1} & 100310.csv & Chromophoric dissolved organic matter absorption at 443 nm\\
\addlinespace
\cellcolor{gray!6}{a\_nap443\_m1} & \cellcolor{gray!6}{m\textsuperscript{-1}} & \cellcolor{gray!6}{100310.csv} & \cellcolor{gray!6}{Non-algal absorption at 443 nm}\\
\addlinespace
station & free text & 100311.csv & Unique ID of the sampled station. Can be used as unique key to merge the data across other files.\\
\addlinespace
\cellcolor{gray!6}{date} & \cellcolor{gray!6}{YYYY-MM-DD} & \cellcolor{gray!6}{100311.csv} & \cellcolor{gray!6}{Date at which the measurement was made}\\
\addlinespace
depth\_m & m & 100311.csv & Depth at which the measurement was made\\
\addlinespace
\cellcolor{gray!6}{longitude} & \cellcolor{gray!6}{Degree decimal} & \cellcolor{gray!6}{100311.csv} & \cellcolor{gray!6}{Longitude of the sampling station}\\
\addlinespace
latitude & Degree decimal & 100311.csv & Latitude of the sampling station\\
\addlinespace
\cellcolor{gray!6}{area} & \cellcolor{gray!6}{free text} & \cellcolor{gray!6}{100311.csv} & \cellcolor{gray!6}{Region where the measurement was made. One of: (1) North Sea, (2) English Channel,  (3) Atlantic Ocean, (4) Med. Sea (Case 2), (5) Adriatic Sea, (6) Baltic Sea}\\
\addlinespace
system & free text & 100311.csv & Dominant river system influencing the site\\
\addlinespace
\cellcolor{gray!6}{gmt\_time} & \cellcolor{gray!6}{Hours decimal} & \cellcolor{gray!6}{100311.csv} & \cellcolor{gray!6}{Representative time of observations at site in Greenwich Mean Time (UTC)}\\
\addlinespace
solar\_zenith\_angle & degree & 100311.csv & Solar zenith angle\\
\addlinespace
\cellcolor{gray!6}{chlorophyll\_a\_mg\_m3} & \cellcolor{gray!6}{mg~m\textsuperscript{-3}} & \cellcolor{gray!6}{100308.csv} & \cellcolor{gray!6}{Chlorophyll-a}\\
\addlinespace
chlorophyll\_b\_mg\_m3 & mg~m\textsuperscript{-3} & 100308.csv & Chlorophyll-b\\
\addlinespace
\cellcolor{gray!6}{chlorophyll\_c\_mg\_m3} & \cellcolor{gray!6}{mg~m\textsuperscript{-3}} & \cellcolor{gray!6}{100308.csv} & \cellcolor{gray!6}{Chlorophyll-c}\\
\addlinespace
pheopigment\_mg\_m3 & mg~m\textsuperscript{-3} & 100308.csv & Pheopigment\\
\addlinespace
\cellcolor{gray!6}{fucoxanthin\_mg\_m3} & \cellcolor{gray!6}{mg~m\textsuperscript{-3}} & \cellcolor{gray!6}{100308.csv} & \cellcolor{gray!6}{Fucoxanthin}\\
\addlinespace
hexanoyloxyfucoxanthin\_19\_mg\_m3 & mg~m\textsuperscript{-3} & 100308.csv & Hexanoyloxyfucoxanthin-19\\
\addlinespace
\cellcolor{gray!6}{butanoyloxyfucoxanthin\_19\_mg\_m3} & \cellcolor{gray!6}{mg~m\textsuperscript{-3}} & \cellcolor{gray!6}{100308.csv} & \cellcolor{gray!6}{Butanoyloxyfucoxanthin-19}\\
\addlinespace
alloxanthin\_mg\_m3 & mg~m\textsuperscript{-3} & 100308.csv & Alloxanthin\\
\addlinespace
\cellcolor{gray!6}{zeaxanthin\_mg\_m3} & \cellcolor{gray!6}{mg~m\textsuperscript{-3}} & \cellcolor{gray!6}{100308.csv} & \cellcolor{gray!6}{Zeaxanthin}\\
\addlinespace
prasixanthin\_mg\_m3 & mg~m\textsuperscript{-3} & 100308.csv & Prasixanthin\\
\addlinespace
\cellcolor{gray!6}{neoxanthin\_mg\_m3} & \cellcolor{gray!6}{mg~m\textsuperscript{-3}} & \cellcolor{gray!6}{100308.csv} & \cellcolor{gray!6}{Neoxanthin}\\
\addlinespace
violaxanthin\_mg\_m3 & mg~m\textsuperscript{-3} & 100308.csv & Violaxanthin\\
\addlinespace
\cellcolor{gray!6}{diatoxanthin\_mg\_m3} & \cellcolor{gray!6}{mg~m\textsuperscript{-3}} & \cellcolor{gray!6}{100308.csv} & \cellcolor{gray!6}{Diatoxanthin}\\
\addlinespace
diadinoxanthin\_mg\_m3 & mg~m\textsuperscript{-3} & 100308.csv & Diadinoxanthin\\
\addlinespace
\cellcolor{gray!6}{peridinin\_mg\_m3} & \cellcolor{gray!6}{mg~m\textsuperscript{-3}} & \cellcolor{gray!6}{100308.csv} & \cellcolor{gray!6}{Peridinin}\\
\addlinespace
carotene\_mg\_m3 & mg~m\textsuperscript{-3} & 100308.csv & Carotene\\
\addlinespace
\cellcolor{gray!6}{lutein\_mg\_m3} & \cellcolor{gray!6}{mg~m\textsuperscript{-3}} & \cellcolor{gray!6}{100308.csv} & \cellcolor{gray!6}{Lutein}\\
\addlinespace
suspended\_particulate\_matter\_g\_m3 & g~m\textsuperscript{-3} & 100306.csv & Suspended particulate matter\\
\addlinespace
\cellcolor{gray!6}{particulate\_organic\_nitrogen\_g\_m3} & \cellcolor{gray!6}{g~m\textsuperscript{-3}} & \cellcolor{gray!6}{100306.csv} & \cellcolor{gray!6}{Particulate organic nitrogen}\\
\addlinespace
total\_particulate\_carbon\_g\_m3 & g~m\textsuperscript{-3} & 100306.csv & Total particulate carbon\\
\addlinespace
\cellcolor{gray!6}{particulate\_organic\_carbon\_g\_m3} & \cellcolor{gray!6}{g~m\textsuperscript{-3}} & \cellcolor{gray!6}{100306.csv} & \cellcolor{gray!6}{Particulate organic carbon}\\
\addlinespace
dissolved\_organic\_carbon\_g\_m3 & g~m\textsuperscript{-3} & 100306.csv & Dissolved organic carbon\\
\addlinespace
\cellcolor{gray!6}{Cast} & \cellcolor{gray!6}{1} & \cellcolor{gray!6}{100312.csv} & \cellcolor{gray!6}{Processed cast number}\\
\addlinespace
Depth & m & 100312.csv & Depth of vertical bin, e.g. -1.00 representing the depth bin [-0.90, -1.10 m]\\
\addlinespace
\cellcolor{gray!6}{TmpWat} & \cellcolor{gray!6}{Degree celsius} & \cellcolor{gray!6}{100312.csv} & \cellcolor{gray!6}{Water temperature}\\
\addlinespace
Cond & ms~cm\textsuperscript{-1} & 100312.csv & Conductivity\\
\addlinespace
\cellcolor{gray!6}{Salin} & \cellcolor{gray!6}{PSU} & \cellcolor{gray!6}{100312.csv} & \cellcolor{gray!6}{Salinity}\\
\addlinespace
SigmaT & 1 & 100312.csv & Quantity (no units) to specify the density of sea water\\
\addlinespace
\cellcolor{gray!6}{TiProf} & \cellcolor{gray!6}{Degree} & \cellcolor{gray!6}{100312.csv} & \cellcolor{gray!6}{Tilt of profiling radiometer}\\
\addlinespace
TiRef & Degree & 100312.csv & Tilt of reference radiometer\\
\addlinespace
\cellcolor{gray!6}{VSpeed} & \cellcolor{gray!6}{m~s\textsuperscript{-1}} & \cellcolor{gray!6}{100312.csv} & \cellcolor{gray!6}{Vertical speed}\\
\addlinespace
Altim & m & 100312.csv & Altimeter sounding of distance from the ocean ground\\
\addlinespace
\cellcolor{gray!6}{N\_OBS} & \cellcolor{gray!6}{1} & \cellcolor{gray!6}{100312.csv} & \cellcolor{gray!6}{Number of observations within depth bin}\\
\addlinespace
EUnnn & W m-2 µm-1 & 100312.csv & In-water upwelling irradiance at wavelength nnn\\
\addlinespace
\cellcolor{gray!6}{EDnnn} & \cellcolor{gray!6}{W m-2 µm-1} & \cellcolor{gray!6}{100312.csv} & \cellcolor{gray!6}{In-water downwelling irradiance at wavelength nnn}\\
\addlinespace
ERnnn & W m-2 µm-1 & 100312.csv & In-air downwelling irradiance at wavelength nnn\\
\addlinespace
\cellcolor{gray!6}{KUnnn} & \cellcolor{gray!6}{m\textsuperscript{-1}} & \cellcolor{gray!6}{100312.csv} & \cellcolor{gray!6}{Diffuse attenuation at wavelength nnn calculated from the upwelling irradiance}\\
\addlinespace
KDnnn & m\textsuperscript{-1} & 100312.csv & Diffuse attenuation at wavelength nnn calculated from the downwelling irradiance\\
\addlinespace
\cellcolor{gray!6}{PAR\_ABS} & \cellcolor{gray!6}{µmol~m\textsuperscript{-2}~s\textsuperscript{-1}} & \cellcolor{gray!6}{100312.csv} & \cellcolor{gray!6}{Phytosynthetically Active Radiation (PAR)}\\
\addlinespace
PAR\%SRF & Percent & 100312.csv & PAR at depth z relative to PAR on the sea surface\\
\addlinespace
\cellcolor{gray!6}{K\_PAR} & \cellcolor{gray!6}{m\textsuperscript{-1}} & \cellcolor{gray!6}{100312.csv} & \cellcolor{gray!6}{Diffuse attenuation for PAR}\\*
\end{longtable}
\endgroup{}


%% The following commands are for the statements about the availability of data sets and/or software code corresponding to the manuscript.
%% It is strongly recommended to make use of these sections in case data sets and/or software code have been part of your research the article is based on.

\codeavailability{TEXT} %% use this section when having only software code available


\dataavailability{TEXT} %% use this section when having only data sets available


\codedataavailability{TEXT} %% use this section when having data sets and software code available


\sampleavailability{TEXT} %% use this section when having geoscientific samples available


\videosupplement{TEXT} %% use this section when having video supplements available


\appendix
\section{}    %% Appendix A

\subsection{}     %% Appendix A1, A2, etc.


\noappendix       %% use this to mark the end of the appendix section. Otherwise the figures might be numbered incorrectly (e.g. 10 instead of 1).

%% Regarding figures and tables in appendices, the following two options are possible depending on your general handling of figures and tables in the manuscript environment:

%% Option 1: If you sorted all figures and tables into the sections of the text, please also sort the appendix figures and appendix tables into the respective appendix sections.
%% They will be correctly named automatically.

%% Option 2: If you put all figures after the reference list, please insert appendix tables and figures after the normal tables and figures.
%% To rename them correctly to A1, A2, etc., please add the following commands in front of them:

\appendixfigures  %% needs to be added in front of appendix figures

\begin{figure}[t]
    \includegraphics[width=12cm]{../../../graphs/appendix01.pdf}
    \caption{My caption}
\end{figure}

\appendixtables   %% needs to be added in front of appendix tables

\begingroup\fontsize{8}{10}\selectfont

\begin{longtable}[t]{>{\raggedright\arraybackslash}p{60em}}
\caption{Scientific articles in peer-reviewed journals using or referencing COASTLOOC}\\
\toprule
Publications\\
\midrule
\endfirsthead
\caption[]{Scientific articles in peer-reviewed journals using or referencing COASTLOOC \textit{(continued)}}\\
\toprule
Publications\\
\midrule
\endhead

\endfoot
\bottomrule
\endlastfoot
\cellcolor{gray!6}{Babin, M., Stramski, D., Ferrari, G. M., Claustre, H., Bricaud, A., Obolensky, G., \& Hoepffner, N. (2003). Variations in the light absorption coefficients of phytoplankton, nonalgal particles, and dissolved organic matter in coastal waters around Europe. Journal of Geophysical Research: Oceans, 108(C7).}\\
\addlinespace
Begouen Demeaux, C., \& Boss, E. (2022). Validation of Remote-Sensing Algorithms for Diffuse Attenuation of Downward Irradiance Using BGC-Argo Floats. Remote Sens. 2022, 14, 4500.\\
\addlinespace
\cellcolor{gray!6}{Belanger, S., Babin, M., \& Larouche, P. (2008). An empirical ocean color algorithm for estimating the contribution of chromophoric dissolved organic matter to total light absorption in optically complex waters. Journal of Geophysical Research: Oceans, 113(C4).}\\
\addlinespace
Beltrán-Abaunza, J. M., Kratzer, S., \& Brockmann, C. (2014). Evaluation of MERIS products from Baltic Sea coastal waters rich in CDOM. Ocean Science, 10(3), 377-396.\\
\addlinespace
\cellcolor{gray!6}{Blix, K., Li, J., Massicotte, P., \& Matsuoka, A. (2019). Developing a new machine-learning algorithm for estimating chlorophyll-a concentration in optically complex waters: A case study for high northern latitude waters by using Sentinel 3 OLCI. Remote Sensing, 11(18), 2076.}\\
\addlinespace
Caillault, K., Roupioz, L., \& Viallefont-Robinet, F. (2021). Modelling of the optical signature of oil slicks at sea for the analysis of multi-and hyperspectral VNIR-SWIR images. Optics Express, 29(12), 18224-18242.\\
\addlinespace
\cellcolor{gray!6}{Chami, M., \& Platel, M. D. (2007). Sensitivity of the retrieval of the inherent optical properties of marine particles in coastal waters to the directional variations and the polarization of the reflectance. Journal of Geophysical Research: Oceans, 112(C5).}\\
\addlinespace
Claustre, H., Fell, F., Oubelkheir, K., Prieur, L., Sciandra, A., Gentili, B., \& Babin, M. (2000). Continuous monitoring of surface optical properties across a geostrophic front: Biogeochemical inferences. Limnology and Oceanography, 45(2), 309-321.\\
\addlinespace
\cellcolor{gray!6}{D'Alimonte, D., Zibordi, G., Kajiyama, T., \& Berthon, J. F. (2014). Comparison between MERIS and regional high-level products in European seas. Remote sensing of environment, 140, 378-395.}\\
\addlinespace
Defoin‐Platel, M., \& Chami, M. (2007). How ambiguous is the inverse problem of ocean color in coastal waters?. Journal of Geophysical Research: Oceans, 112(C3).\\
\addlinespace
\cellcolor{gray!6}{Doerffer, R., \& Schiller, H. (2007). The MERIS Case 2 water algorithm. International Journal of Remote Sensing, 28(3-4), 517-535.}\\
\addlinespace
Doron, M., Babin, M., Mangin, A., \& Hembise, O. (2007). Estimation of light penetration, and horizontal and vertical visibility in oceanic and coastal waters from surface reflectance. Journal of Geophysical Research: Oceans, 112(C6).\\
\addlinespace
\cellcolor{gray!6}{Doron, M., Babin, M., Hembise, O., Mangin, A., \& Garnesson, P. (2011). Ocean transparency from space: Validation of algorithms estimating Secchi depth using MERIS, MODIS and SeaWiFS data. Remote Sensing of Environment, 115(12), 2986-3001.}\\
\addlinespace
Dransfeld, S., Tatnall, A. R., Robinson, I. S., \& Mobley, C. D. (2005). Prioritizing ocean colour channels by neural network input reflectance perturbation. International Journal of Remote Sensing, 26(5), 1043-1048.\\
\addlinespace
\cellcolor{gray!6}{Ferrari, G. M. (2000). The relationship between chromophoric dissolved organic matter and dissolved organic carbon in the European Atlantic coastal area and in the West Mediterranean Sea (Gulf of Lions). Marine Chemistry, 70(4), 339-357.}\\
\addlinespace
Groom, S., Martinez-Vicente, V., Fishwick, J., Tilstone, G., Moore, G., Smyth, T., \& Harbour, D. (2009). The western English Channel observatory: Optical characteristics of station L4. Journal of Marine Systems, 77(3), 278-295.\\
\addlinespace
\cellcolor{gray!6}{Jamet, C., Loisel, H., \& Dessailly, D. (2012). Retrieval of the spectral diffuse attenuation coefficient Kd (lambda) in open and coastal ocean waters using a neural network inversion. Journal of Geophysical Research: Oceans, 117(C10).}\\
\addlinespace
Kratzer, S., \& Moore, G. (2018). Inherent optical properties of the baltic sea in comparison to other seas and oceans. Remote Sensing, 10(3), 418.\\
\addlinespace
\cellcolor{gray!6}{Loisel, H., Stramski, D., Mitchell, B. G., Fell, F., Fournier-Sicre, V., Lemasle, B., \& Babin, M. (2001). Comparison of the ocean inherent optical properties obtained from measurements and inverse modeling. Applied Optics, 40(15), 2384-2397.}\\
\addlinespace
Loisel, H., Vantrepotte, V., Ouillon, S., Ngoc, D. D., Herrmann, M., Tran, V., ... \& Van Nguyen, T. (2017). Assessment and analysis of the chlorophyll-a concentration variability over the Vietnamese coastal waters from the MERIS ocean color sensor (2002–2012). Remote sensing of environment, 190, 217-232.\\
\addlinespace
\cellcolor{gray!6}{Loisel, H., Stramski, D., Dessailly, D., Jamet, C., Li, L., \& Reynolds, R. A. (2018). An inverse model for estimating the optical absorption and backscattering coefficients of seawater from remote‐sensing reflectance over a broad range of oceanic and coastal marine environments. Journal of Geophysical Research: Oceans, 123(3), 2141-2171.}\\
\addlinespace
Matsuoka, A., Hill, V., Huot, Y., Babin, M., \& Bricaud, A. (2011). Seasonal variability in the light absorption properties of western Arctic waters: Parameterization of the individual components of absorption for ocean color applications. Journal of Geophysical Research: Oceans, 116(C2).\\
\addlinespace
\cellcolor{gray!6}{Matsuoka, A., Babin, M., Doxaran, D., Hooker, S. B., Mitchell, B. G., Bélanger, S., \& Bricaud, A. (2014). A synthesis of light absorption properties of the Arctic Ocean: application to semi-analytical estimates of dissolved organic carbon concentrations from space. Biogeosciences, 11(12), 3131-3147.}\\
\addlinespace
Morel, A., \& Bélanger, S. (2006). Improved detection of turbid waters from ocean color sensors information. Remote Sensing of Environment, 102(3-4), 237-249.\\
\addlinespace
\cellcolor{gray!6}{Neukermans, G., Loisel, H., Mériaux, X., Astoreca, R., \& McKee, D. (2012). In situ variability of mass‐specific beam attenuation and backscattering of marine particles with respect to particle size, density, and composition. Limnology and oceanography, 57(1), 124-144.}\\
\addlinespace
Oubelkheir, K., Claustre, H., Bricaud, A., \& Babin, M. (2007). Partitioning total spectral absorption in phytoplankton and colored detrital material contributions. Limnology and Oceanography: Methods, 5(11), 384-395.\\
\addlinespace
\cellcolor{gray!6}{Schroeder, T., Behnert, I., Schaale, M., Fischer, J., \& Doerffer, R. (2007). Atmospheric correction algorithm for MERIS above case‐2 waters. International Journal of Remote Sensing, 28(7), 1469-1486.}\\
\addlinespace
Schroeder, T., Schaale, M., Lovell, J., \& Blondeau-Patissier, D. (2022). An ensemble neural network atmospheric correction for Sentinel-3 OLCI over coastal waters providing inherent model uncertainty estimation and sensor noise propagation. Remote Sensing of Environment, 270, 112848.\\
\addlinespace
\cellcolor{gray!6}{Shahraiyni, T. H., Schaale, M., Fell, F., Fischer, J., Preusker, R., Vatandoust, M., ... \& Tavakoli, A. (2007). Application of the Active Learning Method for the estimation of geophysical variables in the Caspian Sea from satellite ocean colour observations. International Journal of Remote Sensing, 28(20), 4677-4683.}\\
\addlinespace
Tassan, S., Ferrari, G. M., Bricaud, A., \& Babin, M. (2000). Variability of the amplification factor of light absorption by filter-retained aquatic particles in the coastal environment. Journal of Plankton Research, 22(4), 659-668.\\
\addlinespace
\cellcolor{gray!6}{Tassan, S., \& Ferrari, G. M. (2002). A sensitivity analysis of the ‘Transmittance–Reflectance’ method for measuring light absorption by aquatic particles. Journal of Plankton Research, 24(8), 757-774.}\\
\addlinespace
Wei, J., Wang, M., Jiang, L., Yu, X., Mikelsons, K., \& Shen, F. (2021). Global estimation of suspended particulate matter from satellite ocean color imagery. Journal of Geophysical Research: Oceans, 126(8), e2021JC017303.\\
\addlinespace
\cellcolor{gray!6}{Yu, X., Salama, M. S., Shen, F., \& Verhoef, W. (2016). Retrieval of the diffuse attenuation coefficient from GOCI images using the 2SeaColor model: A case study in the Yangtze Estuary. Remote Sensing of Environment, 175, 109-119.}\\
\addlinespace
Yu, X., Lee, Z., Shen, F., Wang, M., Wei, J., Jiang, L., \& Shang, Z. (2019). An empirical algorithm to seamlessly retrieve the concentration of suspended particulate matter from water color across ocean to turbid river mouths. Remote Sensing of Environment, 235, 111491.\\
\addlinespace
\cellcolor{gray!6}{Zhang, T., Fell, F., Liu, Z. S., Preusker, R., Fischer, J., \& He, M. X. (2003). Evaluating the performance of artificial neural network techniques for pigment retrieval from ocean color in Case I waters. Journal of Geophysical Research: Oceans, 108(C9).}\\
\addlinespace
Zhang, T., \& Fell, F. (2004). An approach to improving the retrieval accuracy of oceanic constituents in Case II waters. Journal of Ocean University of China, 3(2), 220-224.\\
\addlinespace
\cellcolor{gray!6}{Zhang, T., \& Fell, F. (2007). An empirical algorithm for determining the diffuse attenuation coefficient Kd in clear and turbid waters from spectral remote sensing reflectance. Limnology and Oceanography: Methods, 5(12), 457-462.}\\
\addlinespace
Zheng, G., \& Stramski, D. (2013). A model based on stacked‐constraints approach for partitioning the light absorption coefficient of seawater into phytoplankton and non‐phytoplankton components. Journal of Geophysical Research: Oceans, 118(4), 2155-2174.\\*
\end{longtable}
\endgroup{}


%% Please add \clearpage between each table and/or figure. Further guidelines on figures and tables can be found below.


\authorcontribution{TEXT} %% this section is mandatory

\competinginterests{TEXT} %% this section is mandatory even if you declare that no competing interests are present

\disclaimer{TEXT} %% optional section

\begin{acknowledgements}
    TEXT
\end{acknowledgements}

%% REFERENCES

%% The reference list is compiled as follows:

\bibliography{/home/filoche/Documents/library.bib}
\bibliographystyle{copernicus}

\end{document}
