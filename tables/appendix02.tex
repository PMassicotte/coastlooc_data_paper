\begingroup\fontsize{8}{10}\selectfont

\begin{longtable}[t]{>{\raggedright\arraybackslash}p{60em}}
\caption{Scientific articles in peer-reviewed journals using or referencing COASTLOOC}\\
\toprule
Publications\\
\midrule
\endfirsthead
\caption[]{Scientific articles in peer-reviewed journals using or referencing COASTLOOC \textit{(continued)}}\\
\toprule
Publications\\
\midrule
\endhead

\endfoot
\bottomrule
\endlastfoot
\cellcolor{gray!6}{Babin, M., Stramski, D., Ferrari, G. M., Claustre, H., Bricaud, A., Obolensky, G., \& Hoepffner, N. (2003). Variations in the light absorption coefficients of phytoplankton, nonalgal particles, and dissolved organic matter in coastal waters around Europe. Journal of Geophysical Research: Oceans, 108(C7).}\\
\addlinespace
Begouen Demeaux, C., \& Boss, E. (2022). Validation of Remote-Sensing Algorithms for Diffuse Attenuation of Downward Irradiance Using BGC-Argo Floats. Remote Sens. 2022, 14, 4500.\\
\addlinespace
\cellcolor{gray!6}{Belanger, S., Babin, M., \& Larouche, P. (2008). An empirical ocean color algorithm for estimating the contribution of chromophoric dissolved organic matter to total light absorption in optically complex waters. Journal of Geophysical Research: Oceans, 113(C4).}\\
\addlinespace
Beltrán-Abaunza, J. M., Kratzer, S., \& Brockmann, C. (2014). Evaluation of MERIS products from Baltic Sea coastal waters rich in CDOM. Ocean Science, 10(3), 377-396.\\
\addlinespace
\cellcolor{gray!6}{Blix, K., Li, J., Massicotte, P., \& Matsuoka, A. (2019). Developing a new machine-learning algorithm for estimating chlorophyll-a concentration in optically complex waters: A case study for high northern latitude waters by using Sentinel 3 OLCI. Remote Sensing, 11(18), 2076.}\\
\addlinespace
Caillault, K., Roupioz, L., \& Viallefont-Robinet, F. (2021). Modelling of the optical signature of oil slicks at sea for the analysis of multi-and hyperspectral VNIR-SWIR images. Optics Express, 29(12), 18224-18242.\\
\addlinespace
\cellcolor{gray!6}{Chami, M., \& Platel, M. D. (2007). Sensitivity of the retrieval of the inherent optical properties of marine particles in coastal waters to the directional variations and the polarization of the reflectance. Journal of Geophysical Research: Oceans, 112(C5).}\\
\addlinespace
Claustre, H., Fell, F., Oubelkheir, K., Prieur, L., Sciandra, A., Gentili, B., \& Babin, M. (2000). Continuous monitoring of surface optical properties across a geostrophic front: Biogeochemical inferences. Limnology and Oceanography, 45(2), 309-321.\\
\addlinespace
\cellcolor{gray!6}{D'Alimonte, D., Zibordi, G., Kajiyama, T., \& Berthon, J. F. (2014). Comparison between MERIS and regional high-level products in European seas. Remote sensing of environment, 140, 378-395.}\\
\addlinespace
Defoin‐Platel, M., \& Chami, M. (2007). How ambiguous is the inverse problem of ocean color in coastal waters?. Journal of Geophysical Research: Oceans, 112(C3).\\
\addlinespace
\cellcolor{gray!6}{Doerffer, R., \& Schiller, H. (2007). The MERIS Case 2 water algorithm. International Journal of Remote Sensing, 28(3-4), 517-535.}\\
\addlinespace
Doron, M., Babin, M., Mangin, A., \& Hembise, O. (2007). Estimation of light penetration, and horizontal and vertical visibility in oceanic and coastal waters from surface reflectance. Journal of Geophysical Research: Oceans, 112(C6).\\
\addlinespace
\cellcolor{gray!6}{Doron, M., Babin, M., Hembise, O., Mangin, A., \& Garnesson, P. (2011). Ocean transparency from space: Validation of algorithms estimating Secchi depth using MERIS, MODIS and SeaWiFS data. Remote Sensing of Environment, 115(12), 2986-3001.}\\
\addlinespace
Dransfeld, S., Tatnall, A. R., Robinson, I. S., \& Mobley, C. D. (2005). Prioritizing ocean colour channels by neural network input reflectance perturbation. International Journal of Remote Sensing, 26(5), 1043-1048.\\
\addlinespace
\cellcolor{gray!6}{Ferrari, G. M. (2000). The relationship between chromophoric dissolved organic matter and dissolved organic carbon in the European Atlantic coastal area and in the West Mediterranean Sea (Gulf of Lions). Marine Chemistry, 70(4), 339-357.}\\
\addlinespace
Groom, S., Martinez-Vicente, V., Fishwick, J., Tilstone, G., Moore, G., Smyth, T., \& Harbour, D. (2009). The western English Channel observatory: Optical characteristics of station L4. Journal of Marine Systems, 77(3), 278-295.\\
\addlinespace
\cellcolor{gray!6}{Jamet, C., Loisel, H., \& Dessailly, D. (2012). Retrieval of the spectral diffuse attenuation coefficient Kd (lambda) in open and coastal ocean waters using a neural network inversion. Journal of Geophysical Research: Oceans, 117(C10).}\\
\addlinespace
Kratzer, S., \& Moore, G. (2018). Inherent optical properties of the baltic sea in comparison to other seas and oceans. Remote Sensing, 10(3), 418.\\
\addlinespace
\cellcolor{gray!6}{Loisel, H., Stramski, D., Mitchell, B. G., Fell, F., Fournier-Sicre, V., Lemasle, B., \& Babin, M. (2001). Comparison of the ocean inherent optical properties obtained from measurements and inverse modeling. Applied Optics, 40(15), 2384-2397.}\\
\addlinespace
Loisel, H., Vantrepotte, V., Ouillon, S., Ngoc, D. D., Herrmann, M., Tran, V., ... \& Van Nguyen, T. (2017). Assessment and analysis of the chlorophyll-a concentration variability over the Vietnamese coastal waters from the MERIS ocean color sensor (2002–2012). Remote sensing of environment, 190, 217-232.\\
\addlinespace
\cellcolor{gray!6}{Loisel, H., Stramski, D., Dessailly, D., Jamet, C., Li, L., \& Reynolds, R. A. (2018). An inverse model for estimating the optical absorption and backscattering coefficients of seawater from remote‐sensing reflectance over a broad range of oceanic and coastal marine environments. Journal of Geophysical Research: Oceans, 123(3), 2141-2171.}\\
\addlinespace
Matsuoka, A., Hill, V., Huot, Y., Babin, M., \& Bricaud, A. (2011). Seasonal variability in the light absorption properties of western Arctic waters: Parameterization of the individual components of absorption for ocean color applications. Journal of Geophysical Research: Oceans, 116(C2).\\
\addlinespace
\cellcolor{gray!6}{Matsuoka, A., Babin, M., Doxaran, D., Hooker, S. B., Mitchell, B. G., Bélanger, S., \& Bricaud, A. (2014). A synthesis of light absorption properties of the Arctic Ocean: application to semi-analytical estimates of dissolved organic carbon concentrations from space. Biogeosciences, 11(12), 3131-3147.}\\
\addlinespace
Morel, A., \& Bélanger, S. (2006). Improved detection of turbid waters from ocean color sensors information. Remote Sensing of Environment, 102(3-4), 237-249.\\
\addlinespace
\cellcolor{gray!6}{Neukermans, G., Loisel, H., Mériaux, X., Astoreca, R., \& McKee, D. (2012). In situ variability of mass‐specific beam attenuation and backscattering of marine particles with respect to particle size, density, and composition. Limnology and oceanography, 57(1), 124-144.}\\
\addlinespace
Oubelkheir, K., Claustre, H., Bricaud, A., \& Babin, M. (2007). Partitioning total spectral absorption in phytoplankton and colored detrital material contributions. Limnology and Oceanography: Methods, 5(11), 384-395.\\
\addlinespace
\cellcolor{gray!6}{Schroeder, T., Behnert, I., Schaale, M., Fischer, J., \& Doerffer, R. (2007). Atmospheric correction algorithm for MERIS above case‐2 waters. International Journal of Remote Sensing, 28(7), 1469-1486.}\\
\addlinespace
Schroeder, T., Schaale, M., Lovell, J., \& Blondeau-Patissier, D. (2022). An ensemble neural network atmospheric correction for Sentinel-3 OLCI over coastal waters providing inherent model uncertainty estimation and sensor noise propagation. Remote Sensing of Environment, 270, 112848.\\
\addlinespace
\cellcolor{gray!6}{Shahraiyni, T. H., Schaale, M., Fell, F., Fischer, J., Preusker, R., Vatandoust, M., ... \& Tavakoli, A. (2007). Application of the Active Learning Method for the estimation of geophysical variables in the Caspian Sea from satellite ocean colour observations. International Journal of Remote Sensing, 28(20), 4677-4683.}\\
\addlinespace
Tassan, S., Ferrari, G. M., Bricaud, A., \& Babin, M. (2000). Variability of the amplification factor of light absorption by filter-retained aquatic particles in the coastal environment. Journal of Plankton Research, 22(4), 659-668.\\
\addlinespace
\cellcolor{gray!6}{Tassan, S., \& Ferrari, G. M. (2002). A sensitivity analysis of the ‘Transmittance–Reflectance’ method for measuring light absorption by aquatic particles. Journal of Plankton Research, 24(8), 757-774.}\\
\addlinespace
Wei, J., Wang, M., Jiang, L., Yu, X., Mikelsons, K., \& Shen, F. (2021). Global estimation of suspended particulate matter from satellite ocean color imagery. Journal of Geophysical Research: Oceans, 126(8), e2021JC017303.\\
\addlinespace
\cellcolor{gray!6}{Yu, X., Salama, M. S., Shen, F., \& Verhoef, W. (2016). Retrieval of the diffuse attenuation coefficient from GOCI images using the 2SeaColor model: A case study in the Yangtze Estuary. Remote Sensing of Environment, 175, 109-119.}\\
\addlinespace
Yu, X., Lee, Z., Shen, F., Wang, M., Wei, J., Jiang, L., \& Shang, Z. (2019). An empirical algorithm to seamlessly retrieve the concentration of suspended particulate matter from water color across ocean to turbid river mouths. Remote Sensing of Environment, 235, 111491.\\
\addlinespace
\cellcolor{gray!6}{Zhang, T., Fell, F., Liu, Z. S., Preusker, R., Fischer, J., \& He, M. X. (2003). Evaluating the performance of artificial neural network techniques for pigment retrieval from ocean color in Case I waters. Journal of Geophysical Research: Oceans, 108(C9).}\\
\addlinespace
Zhang, T., \& Fell, F. (2004). An approach to improving the retrieval accuracy of oceanic constituents in Case II waters. Journal of Ocean University of China, 3(2), 220-224.\\
\addlinespace
\cellcolor{gray!6}{Zhang, T., \& Fell, F. (2007). An empirical algorithm for determining the diffuse attenuation coefficient Kd in clear and turbid waters from spectral remote sensing reflectance. Limnology and Oceanography: Methods, 5(12), 457-462.}\\
\addlinespace
Zheng, G., \& Stramski, D. (2013). A model based on stacked‐constraints approach for partitioning the light absorption coefficient of seawater into phytoplankton and non‐phytoplankton components. Journal of Geophysical Research: Oceans, 118(4), 2155-2174.\\*
\end{longtable}
\endgroup{}
