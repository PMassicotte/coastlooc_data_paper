
\begin{landscape}\begingroup\fontsize{8}{10}\selectfont

\begin{longtable}[t]{>{\raggedright\arraybackslash}p{10em}>{\raggedright\arraybackslash}p{15em}>{\raggedright\arraybackslash}p{8em}>{\raggedright\arraybackslash}p{25em}}
\caption{List of measured parameters}\\
\toprule
Source file & Variable & Units & Description\\
\midrule
\endfirsthead
\caption[]{List of measured parameters \textit{(continued)}}\\
\toprule
Source file & Variable & Units & Description\\
\midrule
\endhead

\endfoot
\bottomrule
\endlastfoot
\cellcolor{gray!6}{absorption.csv} & \cellcolor{gray!6}{wavelength} & \cellcolor{gray!6}{nm} & \cellcolor{gray!6}{}\\
\addlinespace
absorption.csv & a\_p\_m1 & m\textsuperscript{-1} & Total particulate absorption\\
\addlinespace
\cellcolor{gray!6}{absorption.csv} & \cellcolor{gray!6}{a\_nap\_m1} & \cellcolor{gray!6}{m\textsuperscript{-1}} & \cellcolor{gray!6}{Non-algal absorption}\\
\addlinespace
absorption.csv & a\_nap\_adjusted\_m1 & m\textsuperscript{-1} & Non-algal absorption adjusted for...\\
\addlinespace
\cellcolor{gray!6}{absorption.csv} & \cellcolor{gray!6}{a\_cdom\_m1} & \cellcolor{gray!6}{m\textsuperscript{-1}} & \cellcolor{gray!6}{Chromophoric dissolved organig matter absorption}\\
\addlinespace
absorption.csv & a\_cdom\_adjusted\_m1 & m\textsuperscript{-1} & Chromophoric dissolved organig matter absorption with background baseline removed\\
\addlinespace
\cellcolor{gray!6}{absorption.csv} & \cellcolor{gray!6}{a\_phy\_m1} & \cellcolor{gray!6}{m\textsuperscript{-1}} & \cellcolor{gray!6}{Phytoplankton absorption}\\
\addlinespace
absorption.csv & background\_a\_p\_m1 &  & Baseline background of total particulate absorption\\
\addlinespace
\cellcolor{gray!6}{absorption.csv} & \cellcolor{gray!6}{background\_a\_cdom\_m1} & \cellcolor{gray!6}{} & \cellcolor{gray!6}{Baseline background of chromophoric dissolved organig matter absorption}\\
\addlinespace
absorption.csv & background\_a\_nap\_m1 &  & Baseline background of non-algal absorption\\
\addlinespace
\cellcolor{gray!6}{ac9.csv} & \cellcolor{gray!6}{a\_m1} & \cellcolor{gray!6}{m\textsuperscript{-1}} & \cellcolor{gray!6}{Total non-water absorption coefficient}\\
\addlinespace
ac9.csv & c\_m1 & m\textsuperscript{-1} & Total non-water attenuation coefficient\\
\addlinespace
\cellcolor{gray!6}{ac9.csv} & \cellcolor{gray!6}{bp\_m1} & \cellcolor{gray!6}{m\textsuperscript{-1}} & \cellcolor{gray!6}{Particle scattering coefficient}\\
\addlinespace
bathymetry.csv & longitude & Degree decimal & Longitude of the pixel used to extract the bathymetry\\
\addlinespace
\cellcolor{gray!6}{bathymetry.csv} & \cellcolor{gray!6}{latitude} & \cellcolor{gray!6}{Degree decimal} & \cellcolor{gray!6}{Latitude of the pixel used to extract the bathymetry}\\
\addlinespace
bathymetry.csv & bathymetry\_m & m & Bathymetry depth at the sampled stations\\
\addlinespace
\cellcolor{gray!6}{irradiance.csv} & \cellcolor{gray!6}{eu\_w\_m2\_um} & \cellcolor{gray!6}{w~m\textsuperscript{-2}~\textmu m~\textsuperscript{-1}} & \cellcolor{gray!6}{Upward irradiance just beneath the water surface (Eu0-)}\\
\addlinespace
irradiance.csv & ed\_w\_m2\_um & w~m\textsuperscript{-2}~\textmu m~\textsuperscript{-1} & Downward irradiance just beneath the water surface (Ed0-)\\
\addlinespace
\cellcolor{gray!6}{irradiance.csv} & \cellcolor{gray!6}{k\_eu\_m1} & \cellcolor{gray!6}{m\textsuperscript{-1}} & \cellcolor{gray!6}{Attenuation coefficient for upward irradiance (just beneath the water surface)}\\
\addlinespace
irradiance.csv & k\_ed\_m1 & m\textsuperscript{-1} & Attenuation coefficient for downward irradiance (just beneath the water surface)\\
\addlinespace
\cellcolor{gray!6}{reflectance.csv} & \cellcolor{gray!6}{measured\_reflectance\_percent} & \cellcolor{gray!6}{Percent} & \cellcolor{gray!6}{Surface water reflectance}\\
\addlinespace
spectral\_slopes.csv & s\_cdom\_nm1 & nm\textsuperscript{-1} & Spectral slope that describes the approximate exponential decrease in aCDOM\\
\addlinespace
\cellcolor{gray!6}{spectral\_slopes.csv} & \cellcolor{gray!6}{s\_nap\_nm1} & \cellcolor{gray!6}{nm\textsuperscript{-1}} & \cellcolor{gray!6}{Spectral slope that describes the approximate exponential decrease in aNAP}\\
\addlinespace
spectral\_slopes.csv & a\_cdom443\_m1 & m\textsuperscript{-1} & \\
\addlinespace
\cellcolor{gray!6}{spectral\_slopes.csv} & \cellcolor{gray!6}{a\_nap443\_m1} & \cellcolor{gray!6}{m\textsuperscript{-1}} & \cellcolor{gray!6}{}\\
\addlinespace
stations.csv & station &  & Unique ID of the sampled station. Can be used as unique key to merge the data across other files.\\
\addlinespace
\cellcolor{gray!6}{stations.csv} & \cellcolor{gray!6}{date} & \cellcolor{gray!6}{} & \cellcolor{gray!6}{Date at which the measurement was made}\\
\addlinespace
stations.csv & depth\_m & m & Depth at which the measurement was made\\
\addlinespace
\cellcolor{gray!6}{stations.csv} & \cellcolor{gray!6}{longitude} & \cellcolor{gray!6}{Degree decimal} & \cellcolor{gray!6}{Longitude of the sampling station}\\
\addlinespace
stations.csv & latitude & Degree decimal & Latitude of the sampling station\\
\addlinespace
\cellcolor{gray!6}{stations.csv} & \cellcolor{gray!6}{area} & \cellcolor{gray!6}{} & \cellcolor{gray!6}{Region where the measurement was made. One of: (1) North Sea, (2) English Channel,  (3) Atlantic Ocean, (4) Med. Sea (Case 2), (5) Adriatic Sea, (6) Baltic Sea}\\
\addlinespace
stations.csv & system &  & \\
\addlinespace
\cellcolor{gray!6}{stations.csv} & \cellcolor{gray!6}{gmt\_time} & \cellcolor{gray!6}{} & \cellcolor{gray!6}{}\\
\addlinespace
stations.csv & solar\_zenith\_angle & degree & Angle of the sun from the vertical\\
\addlinespace
\cellcolor{gray!6}{pigments.csv} & \cellcolor{gray!6}{chlorophyll\_a\_mg\_m3} & \cellcolor{gray!6}{mg~m\textsuperscript{-3}} & \cellcolor{gray!6}{Chloropyll-a}\\
\addlinespace
pigments.csv & chlorophyll\_b\_mg\_m3 & mg~m\textsuperscript{-3} & Chloropyll-b\\
\addlinespace
\cellcolor{gray!6}{pigments.csv} & \cellcolor{gray!6}{chlorophyll\_c\_mg\_m3} & \cellcolor{gray!6}{mg~m\textsuperscript{-3}} & \cellcolor{gray!6}{Chloropyll-c}\\
\addlinespace
pigments.csv & pheopigment\_mg\_m3 & mg~m\textsuperscript{-3} & Pheopigment\\
\addlinespace
\cellcolor{gray!6}{pigments.csv} & \cellcolor{gray!6}{fucoxanthin\_mg\_m3} & \cellcolor{gray!6}{mg~m\textsuperscript{-3}} & \cellcolor{gray!6}{Fucoxanthin}\\
\addlinespace
pigments.csv & hexanoyloxyfucoxanthin\_19\_mg\_m3 & mg~m\textsuperscript{-3} & Hexanoyloxyfucoxanthin-19\\
\addlinespace
\cellcolor{gray!6}{pigments.csv} & \cellcolor{gray!6}{butanoyloxyfucoxanthin\_19\_mg\_m3} & \cellcolor{gray!6}{mg~m\textsuperscript{-3}} & \cellcolor{gray!6}{Butanoyloxyfucoxanthin-19}\\
\addlinespace
pigments.csv & alloxanthin\_mg\_m3 & mg~m\textsuperscript{-3} & Alloxanthin\\
\addlinespace
\cellcolor{gray!6}{pigments.csv} & \cellcolor{gray!6}{zeaxanthin\_mg\_m3} & \cellcolor{gray!6}{mg~m\textsuperscript{-3}} & \cellcolor{gray!6}{Zeaxanthin}\\
\addlinespace
pigments.csv & prasixanthin\_mg\_m3 & mg~m\textsuperscript{-3} & Prasixanthin\\
\addlinespace
\cellcolor{gray!6}{pigments.csv} & \cellcolor{gray!6}{neoxanthin\_mg\_m3} & \cellcolor{gray!6}{mg~m\textsuperscript{-3}} & \cellcolor{gray!6}{Neoxanthin}\\
\addlinespace
pigments.csv & violaxanthin\_mg\_m3 & mg~m\textsuperscript{-3} & Violaxanthin\\
\addlinespace
\cellcolor{gray!6}{pigments.csv} & \cellcolor{gray!6}{diatoxanthin\_mg\_m3} & \cellcolor{gray!6}{mg~m\textsuperscript{-3}} & \cellcolor{gray!6}{Diatoxanthin}\\
\addlinespace
pigments.csv & diadinoxanthin\_mg\_m3 & mg~m\textsuperscript{-3} & Diadinoxanthin\\
\addlinespace
\cellcolor{gray!6}{pigments.csv} & \cellcolor{gray!6}{peridinin\_mg\_m3} & \cellcolor{gray!6}{mg~m\textsuperscript{-3}} & \cellcolor{gray!6}{Peridinin}\\
\addlinespace
pigments.csv & carotene\_mg\_m3 & mg~m\textsuperscript{-3} & Carotene\\
\addlinespace
\cellcolor{gray!6}{pigments.csv} & \cellcolor{gray!6}{lutein\_mg\_m3} & \cellcolor{gray!6}{mg~m\textsuperscript{-3}} & \cellcolor{gray!6}{Lutein}\\
\addlinespace
carbon.csv & suspended\_particulate\_matter\_g\_m3 & g~m\textsuperscript{-3} & Suspended particulate matter\\
\addlinespace
\cellcolor{gray!6}{carbon.csv} & \cellcolor{gray!6}{particulate\_organic\_nitrogen\_g\_m3} & \cellcolor{gray!6}{g~m\textsuperscript{-3}} & \cellcolor{gray!6}{Particulate organic nitrogen}\\
\addlinespace
carbon.csv & total\_particulate\_carbon\_g\_m3 & g~m\textsuperscript{-3} & Total particulate carbon\\
\addlinespace
\cellcolor{gray!6}{carbon.csv} & \cellcolor{gray!6}{particulate\_organic\_carbon\_g\_m3} & \cellcolor{gray!6}{g~m\textsuperscript{-3}} & \cellcolor{gray!6}{Particulate organic carbon}\\
\addlinespace
carbon.csv & dissolved\_organic\_carbon\_g\_m3 & g~m\textsuperscript{-3} & Dissolved organic carbon\\
\addlinespace
\cellcolor{gray!6}{SPMR} & \cellcolor{gray!6}{Cast} & \cellcolor{gray!6}{1} & \cellcolor{gray!6}{Processed cast number}\\
\addlinespace
SPMR & Depth & m & Depth of vertical bin, e.g. -1.00 representing the depth bin [-0.90, -1.10 m]\\
\addlinespace
\cellcolor{gray!6}{SPMR} & \cellcolor{gray!6}{TmpWat} & \cellcolor{gray!6}{Degree celcious} & \cellcolor{gray!6}{Water temperature}\\
\addlinespace
SPMR & Cond & ms~cm\textsuperscript{-1} & Conductivity\\
\addlinespace
\cellcolor{gray!6}{SPMR} & \cellcolor{gray!6}{Salin} & \cellcolor{gray!6}{PSU} & \cellcolor{gray!6}{Salinity}\\
\addlinespace
SPMR & SigmaT & 1 & Density of sea water\\
\addlinespace
\cellcolor{gray!6}{SPMR} & \cellcolor{gray!6}{TiProf} & \cellcolor{gray!6}{Degree} & \cellcolor{gray!6}{Tilt of profiling radiometer}\\
\addlinespace
SPMR & TiRef & Degree & Tilt of reference radiometer\\
\addlinespace
\cellcolor{gray!6}{SPMR} & \cellcolor{gray!6}{VSpeed} & \cellcolor{gray!6}{m~s\textsuperscript{-1}} & \cellcolor{gray!6}{Vertical speed}\\
\addlinespace
SPMR & Altim & m & Altimeter sounding of distance from the ocean ground\\
\addlinespace
\cellcolor{gray!6}{SPMR} & \cellcolor{gray!6}{N\_OBS} & \cellcolor{gray!6}{1} & \cellcolor{gray!6}{Number of observations within depth bin}\\
\addlinespace
SPMR & EUnnn & w~m\textsuperscript{-2}~\textmu m~\textsuperscript{-1} & In-water upwelling irradiance at wavelength nnn\\
\addlinespace
\cellcolor{gray!6}{SPMR} & \cellcolor{gray!6}{EDnnn} & \cellcolor{gray!6}{w~m\textsuperscript{-2}~\textmu m~\textsuperscript{-1}} & \cellcolor{gray!6}{In-water downwelling irradiance at wavelength nnn}\\
\addlinespace
SPMR & ERnnn & w~m\textsuperscript{-2}~\textmu m~\textsuperscript{-1} & In-air downwelling irradiance at wavelength nnn\\
\addlinespace
\cellcolor{gray!6}{SPMR} & \cellcolor{gray!6}{KUnnn} & \cellcolor{gray!6}{m\textsuperscript{-1}} & \cellcolor{gray!6}{Diffuse attenuation at wavelength nnn calculated from the upwelling irradiance}\\
\addlinespace
SPMR & KDnnn & m\textsuperscript{-1} & Diffuse attenuation at wavelength nnn calculated from the downwelling irradiance\\
\addlinespace
\cellcolor{gray!6}{SPMR} & \cellcolor{gray!6}{PAR\_ABS} & \cellcolor{gray!6}{µmol~m\textsuperscript{-2}~s\textsuperscript{-1}} & \cellcolor{gray!6}{Phytosynthetically Active Radiation (PAR)}\\
\addlinespace
SPMR & PAR\%SRF & Percent & PAR at depth z relative to PAR on the sea surface\\
\addlinespace
\cellcolor{gray!6}{SPMR} & \cellcolor{gray!6}{K\_PAR} & \cellcolor{gray!6}{m\textsuperscript{-1}} & \cellcolor{gray!6}{Diffuse attenuation for PAR}\\*
\end{longtable}
\endgroup{}
\end{landscape}
