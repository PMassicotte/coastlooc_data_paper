\begingroup\fontsize{8}{10}\selectfont

\begin{longtable}[t]{>{\raggedright\arraybackslash}p{18em}>{\raggedright\arraybackslash}p{8em}>{\raggedright\arraybackslash}p{10em}>{\raggedright\arraybackslash}p{25em}}
\caption{List of measured parameters}\\
\toprule
Variable & Units & Source file & Description\\
\midrule
\endfirsthead
\caption[]{List of measured parameters \textit{(continued)}}\\
\toprule
Variable & Units & Source file & Description\\
\midrule
\endhead

\endfoot
\bottomrule
\endlastfoot
\cellcolor{gray!6}{wavelength} & \cellcolor{gray!6}{nm} & \cellcolor{gray!6}{100303.csv} & \cellcolor{gray!6}{}\\
\addlinespace
a\_p\_m1 & m\textsuperscript{-1} & 100303.csv & Total particulate absorption\\
\addlinespace
\cellcolor{gray!6}{a\_nap\_m1} & \cellcolor{gray!6}{m\textsuperscript{-1}} & \cellcolor{gray!6}{100303.csv} & \cellcolor{gray!6}{Non-algal absorption}\\
\addlinespace
a\_nap\_adjusted\_m1 & m\textsuperscript{-1} & 100303.csv & Non-algal absorption adjusted so that baseline background is equal to that of ap\\
\addlinespace
\cellcolor{gray!6}{a\_cdom\_m1} & \cellcolor{gray!6}{m\textsuperscript{-1}} & \cellcolor{gray!6}{100303.csv} & \cellcolor{gray!6}{Chromophoric dissolved organic matter absorption}\\
\addlinespace
a\_cdom\_adjusted\_m1 & m\textsuperscript{-1} & 100303.csv & Chromophoric dissolved organic matter absorption with background baseline removed\\
\addlinespace
\cellcolor{gray!6}{a\_phy\_m1} & \cellcolor{gray!6}{m\textsuperscript{-1}} & \cellcolor{gray!6}{100303.csv} & \cellcolor{gray!6}{Phytoplankton absorption}\\
\addlinespace
background\_a\_p\_m1 & m\textsuperscript{-1} & 100303.csv & Baseline background of total particulate absorption\\
\addlinespace
\cellcolor{gray!6}{background\_a\_cdom\_m1} & \cellcolor{gray!6}{m\textsuperscript{-1}} & \cellcolor{gray!6}{100303.csv} & \cellcolor{gray!6}{Baseline background of chromophoric dissolved organig matter absorption}\\
\addlinespace
background\_a\_nap\_m1 & m\textsuperscript{-1} & 100303.csv & Baseline background of non-algal absorption\\
\addlinespace
\cellcolor{gray!6}{a\_m1} & \cellcolor{gray!6}{m\textsuperscript{-1}} & \cellcolor{gray!6}{100304.csv} & \cellcolor{gray!6}{Total non-water absorption coefficient}\\
\addlinespace
c\_m1 & m\textsuperscript{-1} & 100304.csv & Total non-water attenuation coefficient\\
\addlinespace
\cellcolor{gray!6}{bp\_m1} & \cellcolor{gray!6}{m\textsuperscript{-1}} & \cellcolor{gray!6}{100304.csv} & \cellcolor{gray!6}{Particle scattering coefficient}\\
\addlinespace
longitude & Degree decimal & 100305.csv & Longitude of the pixel used to extract the bathymetry\\
\addlinespace
\cellcolor{gray!6}{latitude} & \cellcolor{gray!6}{Degree decimal} & \cellcolor{gray!6}{100305.csv} & \cellcolor{gray!6}{Latitude of the pixel used to extract the bathymetry}\\
\addlinespace
bathymetry\_m & m & 100305.csv & Bathymetry depth at the sampled stations\\
\addlinespace
\cellcolor{gray!6}{eu\_w\_m2\_um} & \cellcolor{gray!6}{W m-2 µm-1} & \cellcolor{gray!6}{100307.csv} & \cellcolor{gray!6}{Upward irradiance just beneath the water surface (Eu0-)}\\
\addlinespace
ed\_w\_m2\_um & W m-2 µm-1 & 100307.csv & Downward irradiance just beneath the water surface (Ed0-)\\
\addlinespace
\cellcolor{gray!6}{k\_eu\_m1} & \cellcolor{gray!6}{m\textsuperscript{-1}} & \cellcolor{gray!6}{100307.csv} & \cellcolor{gray!6}{Attenuation coefficient for upward irradiance (just beneath the water surface)}\\
\addlinespace
k\_ed\_m1 & m\textsuperscript{-1} & 100307.csv & Attenuation coefficient for downward irradiance (just beneath the water surface)\\
\addlinespace
\cellcolor{gray!6}{measured\_reflectance\_percent} & \cellcolor{gray!6}{Percent} & \cellcolor{gray!6}{100309.csv} & \cellcolor{gray!6}{Surface water reflectance}\\
\addlinespace
s\_cdom\_nm1 & nm\textsuperscript{-1} & 100310.csv & Spectral slope that describes the approximate exponential decrease in aCDOM\\
\addlinespace
\cellcolor{gray!6}{s\_nap\_nm1} & \cellcolor{gray!6}{nm\textsuperscript{-1}} & \cellcolor{gray!6}{100310.csv} & \cellcolor{gray!6}{Spectral slope that describes the approximate exponential decrease in aNAP}\\
\addlinespace
a\_cdom443\_m1 & m\textsuperscript{-1} & 100310.csv & Chromophoric dissolved organic matter absorption at 443 nm\\
\addlinespace
\cellcolor{gray!6}{a\_nap443\_m1} & \cellcolor{gray!6}{m\textsuperscript{-1}} & \cellcolor{gray!6}{100310.csv} & \cellcolor{gray!6}{Non-algal absorption at 443 nm}\\
\addlinespace
station & free text & 100311.csv & Unique ID of the sampled station. Can be used as unique key to merge the data across other files.\\
\addlinespace
\cellcolor{gray!6}{date} & \cellcolor{gray!6}{YYYY-MM-DD} & \cellcolor{gray!6}{100311.csv} & \cellcolor{gray!6}{Date at which the measurement was made}\\
\addlinespace
depth\_m & m & 100311.csv & Depth at which the measurement was made\\
\addlinespace
\cellcolor{gray!6}{longitude} & \cellcolor{gray!6}{Degree decimal} & \cellcolor{gray!6}{100311.csv} & \cellcolor{gray!6}{Longitude of the sampling station}\\
\addlinespace
latitude & Degree decimal & 100311.csv & Latitude of the sampling station\\
\addlinespace
\cellcolor{gray!6}{area} & \cellcolor{gray!6}{free text} & \cellcolor{gray!6}{100311.csv} & \cellcolor{gray!6}{Region where the measurement was made. One of: (1) North Sea, (2) English Channel,  (3) Atlantic Ocean, (4) Med. Sea (Case 2), (5) Adriatic Sea, (6) Baltic Sea}\\
\addlinespace
system & free text & 100311.csv & Dominant river system influencing the site\\
\addlinespace
\cellcolor{gray!6}{gmt\_time} & \cellcolor{gray!6}{Hours decimal} & \cellcolor{gray!6}{100311.csv} & \cellcolor{gray!6}{Representative time of observations at site in Greenwich Mean Time (UTC)}\\
\addlinespace
solar\_zenith\_angle & degree & 100311.csv & Solar zenith angle\\
\addlinespace
\cellcolor{gray!6}{chlorophyll\_a\_mg\_m3} & \cellcolor{gray!6}{mg~m\textsuperscript{-3}} & \cellcolor{gray!6}{100308.csv} & \cellcolor{gray!6}{Chlorophyll-a}\\
\addlinespace
chlorophyll\_b\_mg\_m3 & mg~m\textsuperscript{-3} & 100308.csv & Chlorophyll-b\\
\addlinespace
\cellcolor{gray!6}{chlorophyll\_c\_mg\_m3} & \cellcolor{gray!6}{mg~m\textsuperscript{-3}} & \cellcolor{gray!6}{100308.csv} & \cellcolor{gray!6}{Chlorophyll-c}\\
\addlinespace
pheopigment\_mg\_m3 & mg~m\textsuperscript{-3} & 100308.csv & Pheopigment\\
\addlinespace
\cellcolor{gray!6}{fucoxanthin\_mg\_m3} & \cellcolor{gray!6}{mg~m\textsuperscript{-3}} & \cellcolor{gray!6}{100308.csv} & \cellcolor{gray!6}{Fucoxanthin}\\
\addlinespace
hexanoyloxyfucoxanthin\_19\_mg\_m3 & mg~m\textsuperscript{-3} & 100308.csv & Hexanoyloxyfucoxanthin-19\\
\addlinespace
\cellcolor{gray!6}{butanoyloxyfucoxanthin\_19\_mg\_m3} & \cellcolor{gray!6}{mg~m\textsuperscript{-3}} & \cellcolor{gray!6}{100308.csv} & \cellcolor{gray!6}{Butanoyloxyfucoxanthin-19}\\
\addlinespace
alloxanthin\_mg\_m3 & mg~m\textsuperscript{-3} & 100308.csv & Alloxanthin\\
\addlinespace
\cellcolor{gray!6}{zeaxanthin\_mg\_m3} & \cellcolor{gray!6}{mg~m\textsuperscript{-3}} & \cellcolor{gray!6}{100308.csv} & \cellcolor{gray!6}{Zeaxanthin}\\
\addlinespace
prasixanthin\_mg\_m3 & mg~m\textsuperscript{-3} & 100308.csv & Prasixanthin\\
\addlinespace
\cellcolor{gray!6}{neoxanthin\_mg\_m3} & \cellcolor{gray!6}{mg~m\textsuperscript{-3}} & \cellcolor{gray!6}{100308.csv} & \cellcolor{gray!6}{Neoxanthin}\\
\addlinespace
violaxanthin\_mg\_m3 & mg~m\textsuperscript{-3} & 100308.csv & Violaxanthin\\
\addlinespace
\cellcolor{gray!6}{diatoxanthin\_mg\_m3} & \cellcolor{gray!6}{mg~m\textsuperscript{-3}} & \cellcolor{gray!6}{100308.csv} & \cellcolor{gray!6}{Diatoxanthin}\\
\addlinespace
diadinoxanthin\_mg\_m3 & mg~m\textsuperscript{-3} & 100308.csv & Diadinoxanthin\\
\addlinespace
\cellcolor{gray!6}{peridinin\_mg\_m3} & \cellcolor{gray!6}{mg~m\textsuperscript{-3}} & \cellcolor{gray!6}{100308.csv} & \cellcolor{gray!6}{Peridinin}\\
\addlinespace
carotene\_mg\_m3 & mg~m\textsuperscript{-3} & 100308.csv & Carotene\\
\addlinespace
\cellcolor{gray!6}{lutein\_mg\_m3} & \cellcolor{gray!6}{mg~m\textsuperscript{-3}} & \cellcolor{gray!6}{100308.csv} & \cellcolor{gray!6}{Lutein}\\
\addlinespace
suspended\_particulate\_matter\_g\_m3 & g~m\textsuperscript{-3} & 100306.csv & Suspended particulate matter\\
\addlinespace
\cellcolor{gray!6}{particulate\_organic\_nitrogen\_g\_m3} & \cellcolor{gray!6}{g~m\textsuperscript{-3}} & \cellcolor{gray!6}{100306.csv} & \cellcolor{gray!6}{Particulate organic nitrogen}\\
\addlinespace
total\_particulate\_carbon\_g\_m3 & g~m\textsuperscript{-3} & 100306.csv & Total particulate carbon\\
\addlinespace
\cellcolor{gray!6}{particulate\_organic\_carbon\_g\_m3} & \cellcolor{gray!6}{g~m\textsuperscript{-3}} & \cellcolor{gray!6}{100306.csv} & \cellcolor{gray!6}{Particulate organic carbon}\\
\addlinespace
dissolved\_organic\_carbon\_g\_m3 & g~m\textsuperscript{-3} & 100306.csv & Dissolved organic carbon\\
\addlinespace
\cellcolor{gray!6}{Cast} & \cellcolor{gray!6}{1} & \cellcolor{gray!6}{100312.csv} & \cellcolor{gray!6}{Processed cast number}\\
\addlinespace
Depth & m & 100312.csv & Depth of vertical bin, e.g. -1.00 representing the depth bin [-0.90, -1.10 m]\\
\addlinespace
\cellcolor{gray!6}{TmpWat} & \cellcolor{gray!6}{Degree celsius} & \cellcolor{gray!6}{100312.csv} & \cellcolor{gray!6}{Water temperature}\\
\addlinespace
Cond & ms~cm\textsuperscript{-1} & 100312.csv & Conductivity\\
\addlinespace
\cellcolor{gray!6}{Salin} & \cellcolor{gray!6}{PSU} & \cellcolor{gray!6}{100312.csv} & \cellcolor{gray!6}{Salinity}\\
\addlinespace
SigmaT & 1 & 100312.csv & Quantity (no units) to specify the density of sea water\\
\addlinespace
\cellcolor{gray!6}{TiProf} & \cellcolor{gray!6}{Degree} & \cellcolor{gray!6}{100312.csv} & \cellcolor{gray!6}{Tilt of profiling radiometer}\\
\addlinespace
TiRef & Degree & 100312.csv & Tilt of reference radiometer\\
\addlinespace
\cellcolor{gray!6}{VSpeed} & \cellcolor{gray!6}{m~s\textsuperscript{-1}} & \cellcolor{gray!6}{100312.csv} & \cellcolor{gray!6}{Vertical speed}\\
\addlinespace
Altim & m & 100312.csv & Altimeter sounding of distance from the ocean ground\\
\addlinespace
\cellcolor{gray!6}{N\_OBS} & \cellcolor{gray!6}{1} & \cellcolor{gray!6}{100312.csv} & \cellcolor{gray!6}{Number of observations within depth bin}\\
\addlinespace
EUnnn & W m-2 µm-1 & 100312.csv & In-water upwelling irradiance at wavelength nnn\\
\addlinespace
\cellcolor{gray!6}{EDnnn} & \cellcolor{gray!6}{W m-2 µm-1} & \cellcolor{gray!6}{100312.csv} & \cellcolor{gray!6}{In-water downwelling irradiance at wavelength nnn}\\
\addlinespace
ERnnn & W m-2 µm-1 & 100312.csv & In-air downwelling irradiance at wavelength nnn\\
\addlinespace
\cellcolor{gray!6}{KUnnn} & \cellcolor{gray!6}{m\textsuperscript{-1}} & \cellcolor{gray!6}{100312.csv} & \cellcolor{gray!6}{Diffuse attenuation at wavelength nnn calculated from the upwelling irradiance}\\
\addlinespace
KDnnn & m\textsuperscript{-1} & 100312.csv & Diffuse attenuation at wavelength nnn calculated from the downwelling irradiance\\
\addlinespace
\cellcolor{gray!6}{PAR\_ABS} & \cellcolor{gray!6}{µmol~m\textsuperscript{-2}~s\textsuperscript{-1}} & \cellcolor{gray!6}{100312.csv} & \cellcolor{gray!6}{Phytosynthetically Active Radiation (PAR)}\\
\addlinespace
PAR\%SRF & Percent & 100312.csv & PAR at depth z relative to PAR on the sea surface\\
\addlinespace
\cellcolor{gray!6}{K\_PAR} & \cellcolor{gray!6}{m\textsuperscript{-1}} & \cellcolor{gray!6}{100312.csv} & \cellcolor{gray!6}{Diffuse attenuation for PAR}\\*
\end{longtable}
\endgroup{}
